% NSF proposal generation template style file.
% based on latex stylefiles written by Stefan Llewellyn Smith and
% Sarah Gille, with contributions from other collaborators.
%
\documentclass{proposalnsf}

% See this file for a set of pre-defined journal abbreviations
\newcommand{\jas}{{\it J. Atmos. Sci.}}
\newcommand{\jpo}{{\it J. Phys. Oceanogr.}}
\newcommand{\JPO}{{\it J. Phys. Oceanogr.}}
\newcommand{\jfm}{{\it J. Fluid Mech.}}
\newcommand{\jgr}{{\it J. Geophys. Res.}}
\newcommand{\JGR}{{\it J. Geophys. Res.}}
\newcommand{\jmr}{{\it J. Mar. Res.}}
\newcommand{\arfm}{{\it Ann. Rev. Fluid Mech.}}
\newcommand{\dsr}{{\it Deep-Sea Res.}}
\newcommand{\dao}{{\it Dyn. Atmos. Oceans}}
\newcommand{\jam}{{\it Journal of Applied Meteorology}}
\newcommand{\phfl}{{\it Phys. Fluids}}
\newcommand{\phfla}{{\it Phys. Fluids A}}
\newcommand{\PhilTrans}{{\it Philosophical Transactions of the Royal Society, London}}
\newcommand{\gafd}{{\it Geophys. Astrophys. Fluid Dyn.}}
\newcommand{\gfd}{{\it Geophys. Fluid Dyn.}}
\newcommand{\PCE}{{\it Physics and Chemistry of the Earth}}
\newcommand{\PRL}{{\it Physical Review Letters}}
\newcommand{\ProgOc}{{\it Prog. Oceanography}}
\newcommand{\WHOITR}{Woods Hole Oceanographic Institution Technical Report, WHOI-} 

\newcommand{\degrees}{$\!\!$\char23$\!$}
\renewcommand{\refname}{\centerline{References cited}}

% This handles hanging indents for publications
\def\rrr#1\\{\par
\medskip\hbox{\vbox{\parindent=2em\hsize=6.12in
\hangindent=4em\hangafter=1#1}}}

\def\baselinestretch{1}

\begin{document}

\begin{center}
{\Large{\bf 根據台灣睡眠醫學學會於2013年發表的「全台千人睡眠呼吸中止症大調查」結果發現,近七成(67.9\%)受訪者有打鼾習慣,其中逾三成(34.6\%)為阻塞性睡眠呼吸中止症(Obstructive Sleep Apnea, OSA)的高危險群,經多頻道睡眠檢查(Polysomnography, PSG)診斷為OSA之病患,由於在睡眠期間發生重複性呼吸暫停和過度覺醒而增加了心血管,腦血管和神經認知疾病的發生率。造成OSA的原因有上呼吸道結構異常、呼吸道肌肉過度鬆弛、低醒覺閾值和呼吸驅動力的不穩定等等。而對於OSA病患的治療方式,持續性陽壓呼吸器(Continuous positive airway pressure, CPAP)為第一線也是最有效的治療方式,然而接受度不高(台灣約40\%)及其順從性隨著時間會降低(台灣約64\%, 追蹤9個月),導致治療效果大打折扣。但可以透過甚麼方法增加使用CPAP的遵從性,應是每個睡眠中心都積極在做的事。根據文獻及實證醫學的證據,個案管理師配戴CPAP前的衛教及動機加強的支持是最有效的方式,再來是遠端監測隨時提供必要的解決方案,或是耳鼻喉科鼻腔阻力的處理,復健科的口腔機能訓練,身心科的失眠及焦慮治療,牙科方面若因CPAP 壓力過高而無法耐受者,CPAP 與牙套的合併使用,則可大幅降低所需壓力,增加患者對 CPAP 的接受度等等都是有效的方式。本研究希望能建立全人醫療治療模式,提供CPAP的全方位整合治療,提高Acceptance, Compliance, Adherence進而提高治療成功率,防範共病症及併發症的發生。}}\\*[3mm]
{\bf Holistic Health Care for Sleep Hygiene Education} \\*[3mm]

PI Names \\
More PI Names

\end{center}





\noindent
{\bf Intellectual Merit}

\ \\

\noindent
{\bf Broader Impacts}

\renewcommand{\thepage} {B--\arabic{page}}

\newpage

% reset page numbering to 1.  This is helpful, since the text can only
% be 15 pages, and reviewers will want to believe we've kept within
% those limits

\pagenumbering{arabic}
\renewcommand{\thepage} {D--\arabic{page}}

\newpage

\centerline{\bf Results from Prior NSF Support}

\noindent
{\bf Previous Award Title}
{\it award number} (PI); dates, \$amount

Research carried out ....

\ \\

\noindent{\Large \bf PROJECT DESCRIPTION}

\section{Introduction}

阻塞性睡眠呼吸中止症(OSA),會導致睡眠中短暫覺醒或血中氧氣不足。大聲打鼾是OSA的典型特徵,患者間歇性缺氧連續發生,會使病人在覺醒和入睡之間擺盪,睡眠的連續性受干擾而片斷化,讓睡覺無法安穩,導致白天精神不濟、疲累嗜睡、無法專注等。也會因交感神經活性增高,引發血壓升高、心律不整,伴隨心血管疾病的風險增加:如中風或心肌梗塞等。持續性陽壓呼吸器(Continuous positive airway pressure, CPAP)自1981年首次發表用來治療OSA以來,迅速成為首選療法。直到今日,它仍然是大多數睡眠醫師的首選治療方法,主要是因為它在防止上呼吸道阻塞方面的效果很高。CPAP作用原理是讓你在睡眠時,先戴上適合自己臉鼻形的密封式鼻面罩或是口鼻面罩,此面罩連著一條蛇形管,另一端接到呼吸器本身,在設定壓力參數後,可以提供上呼吸道一正壓,當你在吸氣時,此正壓空氣會將塌陷的咽喉氣道重新撐開,這樣不但可以改善打鼾,連帶也改善呼吸中止引起的低血氧症,也就是讓你在睡夢中不用再缺氧了,所以CPAP可以降低OSA引起的心血管疾病,改善白天打瞌睡的現象。但是,並非所有患者都能接受或能忍受CPAP治療。文獻指出即使罹患中重度的OSA患者,高達29\%至83\%的患者無法接受CPAP治療或患者會自覺不舒服,在睡夢中會移除呼吸器面罩(順從性compliance的定義為每晚平均使用時間>4小時)。而台灣CPAP的接受度(Acceptance)不高約40\%,且順從性(Compliance)隨著時間也會降低在台灣約64\%,至於遵從性(Adherence, 追蹤時仍規則在使用的比率)方面,以雙和醫院兩年以上(862人)的追蹤約33\%,也就是長期追蹤下來,CPAP的有效治療率大概只剩一成左右,這是睡眠團隊還必須努力的理由。

\section{Research Purpose}

從 1965 年開始注意到OSA的存在後,經過胸腔科、耳鼻喉科、神經科、精神科、牙科等共同努力,治療OSA的方法主要分成內科治療及外科治療。內科治療方面CPAP是治療OSA的優先選擇,成功率幾近90%,是中度到重度 OSA病人最主要的治療方法,因其高效率的症狀控制(使用當天即可降低AHI到幾近正常值),改善生活品質(記憶力、注意力、及行政效能)及減輕呼吸中止症所造成的結果(降低血壓,降低心血管的發生率,以及使用CPAP後2-7天即可改善駕駛的行為,避免交通意外事故的發生)。且許多研究顯示 OSA 病人接受 CPAP治療後的好處很多,包括改善症狀;減少床旁伴侶的睡眠障礙或者生活品質;減少心臟血管疾病、神經認知功能退化的危險性,或者因為睡眠呼吸中止症增加的死亡率;以及減少車禍意外事故發生。但根據文獻資料,CPAP的接受度(Acceptance)頂多50\%,至於遵從性(Adherence)大約只有35\%,而順從性(Compliance)約60\%,本研究計畫欲配合多科整合(multimodality)提供一個創新的醫療服務模式,建立CPAP的全方位整合治療(integrated therapy),提高台灣OSA患者使用陽壓呼吸器的Acceptance, Compliance, Adherence rate。

\section{Material and Methods}

\subsection{方法、流程}
組織架構: 

睡眠內科醫師(胸內、神內、身心、復健、減重內科)、睡眠外科醫師(耳鼻喉、牙科、減重外科)、睡眠技師、個管師、心理師、營養師等。

計畫方法與流程:

萬芳醫院於2020年接受睡眠檢查人數為1399人,接受CPAP titration人數為234人,最後接受耳鼻喉科手術為24人,若以CPAP接受率40\%計算,故本計劃預定在 12 個月內收案 80 人。於臨床上懷疑OSA的成年病患(20 到 70 歲),安排睡眠檢查(Polysomnography, PSG)診斷為OSA後,至耳鼻喉科接受上呼吸道內視鏡動態評估(Awake endoscopy with muller)檢查及牙科側顱術 ( Cephalometry ) 檢查,病患如果為中重度睡眠呼吸中止症或輕度但合併白天嗜睡或有心血管共病症者,即應建議接受CPAP治療,本計畫病患均遵循 OSA 衛教(適度運動、生活作息及飲食調整等),並且加入多科整合模式(睡眠技師滴定壓力、個管師衛教及追蹤、耳鼻喉科醫師評估鼻腔是否需要介入、牙科醫師評估牙套是否需要介入、復健科醫師評估肌功能訓練是否需要介入、神內身心科醫師評估CBTi是否需要介入),於一周、兩周、一個月、三個月、六個月及十二個月治療期間定期追蹤,看能否有效提升阻塞型睡眠呼吸中止症病人對CPAP的接受度、順從度、遵從率。



參與人員:

整個睡眠醫療團隊(各科醫師、睡眠技師、個管師、心理師、營養師等)及病患的參與跟家屬的支持(尤其是枕邊人)。

\subsection{教育訓練、會議、成效評估、評量指標、統計}

equipment:
雷射針灸儀 RJ Laserpen,我正在使用的 500mW 低能量雷射 810nm,穿透皮膚 2 cm 進入穴道
睡眠醫學 LED能量帽、雷射針灸(內關、神門)等
搭配能量帽,幫助門診睡眠障礙的病患。(在我的全人睡眠照護計畫中)
十萬元研究經費,先讓病患借用(收押金),評估成效。然後他們喜歡滿意,也許會想自己買一頂。
中醫通鼻藥水

睡眠障礙整合門診,週三下午祁力行在中醫科,雷射針灸能幫助提升睡眠品質。
—周珊如 anti-snoring guard

每個月的睡眠中心會議及睡眠外科會議,討論各種治療方式的優缺點及如何整合,給不同表現型的OSA患者提供最適合的治療方式。

現今有效提升OSA病患對CPAP的接受度及順從度的方法不外乎: (1)治療前衛教使用CPAP的重要性:使用前由睡眠專業人員解說,確保了解機器的使用方式、治療方法、常見問題以及機器如何幫助緩解睡眠呼吸中止症狀。並且需教導患者未經治療的OSA,可能會對的生活質量和健康造成許多不好的影響,如:睡覺容易醒來、頻尿、白天精神較差等。(2)選擇合適的面罩:許多患者由於未使用適合的面罩而造成遵從性不佳,因此需要對幾種面罩進行試驗,找到可以舒適使用的面罩。此外,不要強迫立即整夜佩戴面罩,可鼓勵練習在醒著或小睡時戴上面罩,並漸進式增加每晚戴面罩的時間,這將有助於患者減輕戴上面罩的不適感。
(一)鼻枕:適用於沒有上唇牙齒支持或幽閉恐懼症的患者。
(二)鼻罩:遮蓋鼻子,並且比全面罩更輕巧。
(三)全面罩:覆蓋口鼻部,適用在治療期間出現嚴重鼻塞或張口呼吸的患者。

(3)加裝潮濕器:加熱加濕可解決常見的口乾舌燥的情形,並使佩戴面罩時更加舒適。一些患者可能在睡覺時型習慣性張口呼吸,CPAP可能會使口乾惡化,因此使用下巴帶可以幫助嘴巴保持閉合且可減少漏氣量。(4)善用緩衝功能(RAMP) :大多數機器都允許患者啟動這個選項,壓力從預設壓力開始,在設置的緩衝時間內,緩慢增加到治療壓力。緩衝功能有足夠時間讓患者習慣壓力,且可在醒來幫助新入睡。(5)彈性的呼氣壓力(吐氣降壓):通過降低呼氣開始時的壓力,軟化從吸氣到呼氣的壓力,使呼氣時較不會感覺到阻力。此功能在每次呼吸時提供適量的壓力,有助於增加患者的遵從性,不同的機器提供不同的彈性的呼氣壓力功能。(6)保持機器、面罩和管路清潔:將清潔作為日常工作的一部分,因未經清洗的面罩和管路可能導致患者鼻塞,感冒和細菌,這些問題可能會對CPAP治療產生負面影響,並降低患者使用機器的遵從性。所以有無個管師的介入是非常重要的部分(Motivational enhancement) 。

外科治療(Upper airway soft tissue surgery, Orthognathic surgery, Bariatric surgery)的成效評估為治療後三到六個月追蹤客觀睡眠檢查及病患主觀症狀改善的結果,當然每年的回診追蹤也是必需的。

內科治療(CPAP, Oral appliance, Myofunctional therapy, lifestyle modification)的成效評估為治療後規則回診追蹤治療的遵從度及病患主觀症狀改善的結果。

用以評估療效之參數為睡眠檢查及睡眠相關問卷及CPAP的使用資料:

(1) 多頻道睡眠檢查(PSG): 呼吸中止及低下指數(AHI)、氧氣飽和度下降指數(ODI)、最低血氧飽和度(LSAT)及血氧濃度小於90\%的時間(CT90) 。 
「廠商個資保密協定」。
不知 CPAP
home 睡眠監測 有沒有個資疑慮呢?謝謝

-睡眠中心in lab PSG 與臨床資料,收集進行 deep learning (已發展成熟的 AI model)來預測、建議 OSA 病患的最佳治療方案。https://www.nature.com/articles/s41598-020-62223-4

-居家睡眠機器收集到的資料,以及臨床病歷,都是很有價值的分析 database。

(2) 艾普沃斯嗜睡量表(Epworth Sleepiness Scale, ESS): 用來評估日間嗜睡狀況,並且給予量化評分 。

(3) 匹茲堡睡眠品質量表(Pittsburgh Sleep Quality Index, PSQI):國際上對於睡眠一般品質好壞的評估,利用評分的高低區分睡眠品質的好壞

(4) 失眠嚴重度指標 (Insomnia Severity Index): 用來評估夜間睡眠品質狀況,並且給予量化評分 。

(5) 每三個月讀取CPAP報告: 監測Adherence rate and Compliance rate。



比較 holistic sleep hygiene education 全人教育的功效



\subsection{Statistics}

以描述性研究分析個案基本資料,並以 ANOVA 和 Paired t-test 分析OSA個案基本資料、睡眠問卷、睡眠檢查相關指數及CPAP使用報告的結果,預期multimodality and integrated therapy能有效提升阻塞型睡眠呼吸中止症病人對CPAP的接受度、順從度、遵從率。



\section{Time Line and Management Plan}

 第1月 第2月 第3月 第4月 第5月 第6月 第7月 第8月 第9月 第10月 第11月 第12月

文獻收集 ●     ●     ● 

軟硬體設置 ●     ● 

研究資料蒐集 ●     ●      ●     ●     ●     ●      ●     ● 

資料庫設立及建置 ●      ●     ●      ● 

研究分析及報告                                                                   ●      ●      ●

\section{Summary:  Significance of proposed work}

results announcement and publication

%\subsection{Intellectual Merit}

%\subsection{Broader Impacts}



\newpage
\pagenumbering{arabic}
\renewcommand{\thepage} {E--\arabic{page}}

\bibliography{draft}
\bibliographystyle{jponew}

\newpage
\pagenumbering{arabic}
\renewcommand{\thepage} {G--\arabic{page}}
\noindent{\Large \bf BUDGET JUSTIFICATION}

\end{document}
